%%%%%%%%%%%%%%%%%%%%%%%%%%%%%%%%%%%%%%%%%
% Plain Cover Letter
% LaTeX Template
%
% This template has been downloaded from:
% http://www.latextemplates.com
%
% Original author:
% Rensselaer Polytechnic Institute (http://www.rpi.edu/dept/arc/training/latex/resumes/)
%
%%%%%%%%%%%%%%%%%%%%%%%%%%%%%%%%%%%%%%%%%

%----------------------------------------------------------------------------------------
%       PACKAGES AND OTHER DOCUMENT CONFIGURATIONS
%----------------------------------------------------------------------------------------

\documentclass[11pt]{letter} % Default font size of the document, change to 10pt to fit more text
\usepackage{graphicx}
%\usepackage{newcent} % Default font is the New Century Schoolbook PostScript font
%\usepackage{helvet} % Uncomment this (while commenting the above line) to use the Helvetica font

% Margins
\usepackage[left=1.25in,right=1.25in,top=1in,bottom=0.5in]{geometry}
%\let\raggedleft\raggedright % Pushes the date (at the top) to the left, comment this line to have the date on the right

\usepackage{eso-pic,graphicx}
 \begin{document}

%----------------------------------------------------------------------------------------
%       ADDRESSEE SECTION
%----------------------------------------------------------------------------------------

\begin{letter}{}

%----------------------------------------------------------------------------------------
%       YOUR NAME & ADDRESS SECTION
%----------------------------------------------------------------------------------------

\address{Jin Whan Bae\\
nuclearbae@gmail.com\\
1 Bethel Valley Rd.\\
Oak Ridge, TN 37830}


%----------------------------------------------------------------------------------------
%       LETTER CONTENT SECTION
%----------------------------------------------------------------------------------------

\opening{Professor Sara A. Pozzi,}

Please find enclosed a manuscript entitled: ``Deep Learning Approach to Nuclear Fuel Transmutation in a Fuel
Cycle Simulator'' which I and my coauthors, Rykhlevskii, Chee, and Huff, 
are submitting for exclusive consideration for publication as a research 
article in Annals of Nuclear Energy.

This manuscript demonstrates an effective use of neural network regression
models to predict depleted fuel composition in a nuclear fuel cycle
simulator. The paper describes the workflow in detail enough for this
work to be reproduced.

This work extends a previous concept of training a neural network to
predict complex calculations, by quantifying the effects of using
neural networks in a fuel cycle simulator. The field of system-level
fuel cycle modeling has always tried to find a balance between
fidelity, flexibility, and quick computation time. A well-designed
neural network trained on data from high-fidelity calculations, can
be utilized to model a fuel cycle facility, without a large
computational burden. This workflow of using data from a complex
modeling tool (e.g. SCALE) to train a neural network model 
for implementation in a fuel cycle simulator, as demonstrated in
this work, is promising.


Thank you for your consideration of this work. I expect it will be of interest
to a broad readership concerned with the nuclear fuel cycle, used
nuclear fuel management, and applications of machine learning in nuclear engineering applications.  Please address correspondence concerning
this manuscript to me at the Oak Ridge National Laboratory.

\closing{Sincere regards,
\fromsig{Jin Whan Bae\\Research Associate\\Oak Ridge National Laboratory\\nuclearbae@gmail.com}
}

%----------------------------------------------------------------------------------------

\end{letter}

\end{document}


