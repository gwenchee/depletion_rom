\section{Future Work}

A trained model is only as good as the data it is trained on.
This work can be improved and expanded by generating
more comprehensive depletion data that covers a wider
range of enrichment and burnup ranges. An automation
script might run SCALE ORIGEN to perform depletion calculations
for a wide range of enrichment (e.g. 0.7 - 4.99 wt\%) and burnup (e.g. 0 - 80,000 MWd/MT),
for a single assembly design. Assumptions of criticality
and irradiation time should be made as well. The results
could then be parsed into a csv file and stored for
the training of a new model. This will allow better
prediction of the model for higher burnups and `fringe'
burnup-enrichment assemblies.

Methods used in this paper have the potential to be expanded into more
complicated problems. For example, the method
could be applied to \gls{MOX} fuel depletion, taking
into account varying uranium and plutonium concentrations.
However, a \gls{UDB} equivalent is not available
for training a \gls{MOX} model, so a
data-generating process, consisting of a high-fidelity
depletion calculation of randomized \gls{MOX} fuel
compositions within acceptable ranges of reactor
criticality, similar to the process mentioned above,
will have to take place. After the data
is generated, the neural network model can be trained
to predict \gls{MOX} depletion.

Another interesting application of this method is for
\gls{MSR} system optimization. Current work on
\glspl{MSR} includes of optimizing non-core operating
parameters such as reprocessing scheme and flow rate.
Fuel transmutation calculations
can be usually computationally burdensome for \gls{MSR}
simulations, since the flowing fuel is depleted and
reprocessed continuously. \gls{MSR} models implement semi-continuous
methods in which the depletion-to-reprocessing time is
very short (usually 3 days), which makes an
\gls{MSR} lifetime simulation (if 60 years)
require $\sim 7,300$ depletion calculations.
The computational burden
makes it impossible to use brute-force methods,
such as grid search of all possible parameters.
However, if a quick depletion calculation becomes possible
with a well-trained prediction model, the
computational burden will dramatically decrease.
However, problems with generating enough
training data, accuracy, and the model's ability to
extrapolate remain.