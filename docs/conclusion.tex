\section{Conclusion}
This work shows that fuel cycle analyses can be improved
by using a neural network predictive model for depletion.
The neural network model predicted the \gls{UNF} inventory
with less than 5\% error for individual isotopes,
less than 1\% error for waste profile 
and less than 2\% error for the fissile quality.
The predictive model outperformed the average recipe
method in every metric.

This work also shows that quick, open-source depletion models
can be implemented using prediction models, from
complex, inaccessible depletion algorithms and
datasets. \gls{NFC} simulators struggle to find a balance
between fidelity and rapidity. Using high-fidelity
models become prohibitively computationally expensive
since \gls{NFC} simulators may need to run
hundreds of depletion calculations for multiple
facilities. On the other hand, using simpler methods
like recipes are overly simplified solutions
that do not take into account variations in fuel
parameters such as burnup.
A well-trained predictive algorithm can find middle
ground between rapidity and fidelity.

We cannot avoid the criticism that the model is validated
against the dataset used to train the model. However, the purpose
of this work is to create a model that can quickly reproduce the
database without having to distribute the database, which is proprietary
and large in size. The pickled file that contains
the model and data scaling objects is only 38.2 kB, meaning that it
can be easily distributed and imported in external software.

[cringe]

It is noteworthy that there are not a lot of accessible data in the
nuclear engineering discipline. Large-scale depletion
calculation results like the \gls{UDB} are exceptional
but rare cases. The value of data is exponentially increasing,
with the advancement of data science and machine learning.
For the long-term advancement of the field, maybe the
community should encourage collection and storage of more
data and simulation results in
a central repository, for the betterment of the field.

[cringe]

