\section{Conclusion and Discussion}

This work shows that depleted fuel composition predictions
in fuel cycle simulators can be improved
by using a predictive neural network  model.
The neural network model predicted the \gls{UNF} inventory
with less than 5\% error for important isotopes,
less than 1\% error for waste management profile metrics, 
and less than 0.1\% error for \textsuperscript{239}Pu equivalence.
The predictive model outperformed the average recipe
method in every metric.

We implemented this model in \Cyclus to provide a
 reactor model with dynamic reactor parameters,
which can simulate potential future improvement scenarios
in reactor operation.

This work also shows that open-source depletion models
that run quickly
can be implemented using prediction models, from
complex depletion algorithms and
datasets. \gls{NFC} simulators struggle to find a balance
between fidelity and rapidity. Using a high-fidelity
model is prohibitively computationally expensive
since \gls{NFC} simulators may need to run
hundreds of depletion calculations for multiple
facilities. On the other hand, using simpler methods
like recipes may be too simple for some applications
since they do not take into account variations in fuel
parameters such as burnup.
A well-trained predictive algorithm can find a middle
ground between rapidity and fidelity.

Ideally, the model would be validated against an external
dataset, instead of the dataset used to train the model.
 However, the purpose
of this work is to create a model that can quickly reproduce the
database without having access to the database, which is private
data
and large in size. The pickled file that contains
the model and data scaling objects is only 38.2 kB, meaning that it
can be easily distributed and imported in external software, without
revealing detailed information about the actual dataset.

An accessible, large-scale depletion database 
like the \gls{UDB} are valuable
but rare. The value of data is increasing,
with the advancement of data science and machine learning.
For the long-term advancement of the field, the
community should encourage collection and storage of more
data and simulation results in
a central repository.

